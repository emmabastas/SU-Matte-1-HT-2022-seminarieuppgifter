\documentclass{article}

\usepackage{amsmath}
\usepackage{amsthm}
\usepackage{mathtools}

\usepackage{mdframed}
\usepackage{lipsum}

\newmdtheoremenv{lemma}{Theorem}
\renewcommand{\thefootnote}{\fnsymbol{footnote}}

\title{Seminarieuppgift 4 - Absolutbelopp}
\author{Emma Bastås}

\begin{document}

\maketitle

\noindent Uppgiften är att finna de värden på $x$ som upfyller olikheten:

\begin{gather*}
  2|x+2| < |x| + |x+3|\text{.} \label{*}\tag{$\star$}
\end{gather*}

\noindent Vi adderar $2|x+2|$ till båda led och skriver sedan om uttrycket med hjälp av identiteten $|x| = \sqrt{x^{2}}$:

\begin{align*}
  \text{(\ref{*})} \iff& \quad 0 < |x| + |x + 3| - 2|x+2| \\
  \iff& \quad 0 < \sqrt{x^{2}} + \sqrt{(x + 3)^{2}} - 2 \sqrt{(x + 2)^{2}}\text{.} \label{**}\tag{$\star\star$}
\end{align*}
\noindent Låt $f(x)$ definieras till högerledet i uttrycket (\ref{**}):

\begin{gather*}
  f(x) =  \sqrt{x^{2}} + \sqrt{(x + 3)^{2}} - 2 \sqrt{(x + 2)^{2}}\text{.}
\end{gather*}

\noindent Uppgiften omformuleras nu till denna ekvivalenta problemställning:
\\
\\
\centerline{\emph{Finn alla värden på $x$ så att $f(x) > 0$.}}
\\
\\
Nu finner vi alla nollställen till $f(x)$, vi börjar med att ställa upp ekvationen och flytta över en av rötterna till högerledet:

\begin{align}
  \quad & f(x) = 0 \nonumber\\
  \iff \quad &\sqrt{x^{2}} + \sqrt{(x + 3)^{2}} - 2 \sqrt{(x + 2)^{2}} = 0 \nonumber\\
  \iff \quad &\sqrt{x^{2}} + \sqrt{(x + 3)^{2}} = 2 \sqrt{(x + 2)^{2}}\text{.} \label{initial}
\end{align}
\\
Båda led kvadreras varefter vi förenklar så att de kvarvarande kvadratrötterna står ensam i högerledet:

\begin{align}
  \text{(\ref{initial})} \implies \quad & \biggl(\sqrt{x^{2}} + \sqrt{(x + 3)^{2}}\biggr)^{2} = \biggl(2 \sqrt{(x + 2)^{2}}\biggr)^{2}\text{.} \nonumber\\
  \iff \quad & x^{2} + 2\sqrt{x^{2}}\sqrt{(x+3)^{2}} + (x + 3)^{2} = 4(x+2)^{2} \nonumber\\
  \iff \quad & 2x^{2} + 10x + 7 = 2\sqrt{x^{2}}\sqrt{(x+3)^{2}}\text{.} \label{squared}
\end{align}
\\
Återigen kvadreras båda led och vi förenklar:

\begin{align}
  \text{(\ref{squared})} \implies \quad & (2x^{2} + 10x + 7)^{2} = (2\sqrt{x^{2}}\sqrt{(x+3)^{2}})^{2} \nonumber\\
  \iff \quad & 4x^{4} + 40x^{3} + 128x^{2} + 140x + 49 = 4x^{4} + 24x^{3} + 36x^{2} \nonumber\\
  \iff \quad & 16x^{3} + 92x^{2} + 140x + 49 = 0\text{.} \label{3rd_degree}
\end{align}
\\
Nu har vi kört fast, visserligen går det att lösa tredjegradsekvationer av denna karaktär, men det ligger utanför denna seminariekurs. Om vi blickar tillbaka till ekv (\ref{squared}) ser vi räddningen. Vi kan återigen använda identiteten $|x| = \sqrt{x^{2}}$ på (\ref{squared}):

\begin{align}
  \text(\ref{squared}) \iff 2x^{2} + 10x + 7 &= 2\sqrt{x^{2}}\sqrt{(x+3)^{2}}\nonumber\\
                        &= 2\sqrt{(x(x+3))^{2}}\nonumber\\
                        &= 2|x(x+3)|\nonumber\\
                        &= 2|x^{2} + 3x|\text{.} \label{single_abs}
\end{align}
\\
Nu begränsar vi oss till fallet då $x^{2} + 3x \geq 0$ och löser den resulterande ekvationen:

\begin{align*}
  \text{(\ref{single_abs})} \iff \quad & 2x^{2} + 10x + 7 = 2x^{2} + 6x\\
  \iff \quad & 4x + 7 = 0 \\
  \iff \quad & x = -\frac{7}{4}\text{.}
\end{align*}
\\
Vi ser att $x = -\frac{7}{4}$ löser ekv (\ref{squared}) vilket innebär att detta också är en lösning för (\ref{3rd_degree}) eftersom den medförs av (\ref{squared}). Faktorsatsen ger nu att $(x + \frac{7}{4})$ är en faktor till (\ref{3rd_degree}) och vi kan således utföra polynomdivison. Resultaten av en sådan division ger:

\begin{gather}
  \text{(\ref{3rd_degree})} \iff (x + 7 / 4)(16x^{2} + 64x + 28) = 0\text{.} \label{2nd_degree}
\end{gather}
\\
De resterande lösningarna till (\ref{3rd_degree}) är nollställerna till andragradspolynomet i (\ref{2nd_degree}), vi löser denna andragradsekvation med valfi metod, lösningarna blir:

\begin{gather*}
  x = -\frac{7}{2} \quad \text{och} \quad x = - \frac{1}{2}\text{.}
\end{gather*}
\\
$f(x) = 0$ har medfört ekv (\ref{2nd_degree}) som har de tre lösningarna $x = -7/2$, $x = -7/4$ och $x = -1/2$, vi testar nu all tre värden $f(x) = 0$ och finner att endast $x = -1/2$ är en lösning till $f(x) = 0$.
\\
\\
Vi väljer nu två konkreta tal $a$ och $b$ som ligger till vänster respektive höger om nollstället, säg $a = -2$ och $b = 0$. Vi finner att $f(a) = 3$ och $f(b) = -1$. Nu har vi faktiskt all information vi behöver för att visa att $f < -1/2$ är lösningen till (\ref{**}):
\\
\\
$f(x)$ är en kontinuerlig funktion och $x = -1/2$ är dess enda nollstället. Det finns en punkt i intervallet $]-\infty, -1/2[$, nämligen $a$ så att $f(a) > 0$. Om det finns ett $\mu$ i detta intervall så att $f(\mu) < 0$ så säger sats 14 i Persson och Böiers \emph{Analys i en variabel}\footnote{Upplaga 3:2 s.153} att $f(x)$ då antar alla värden mellan $f(a)$ och $f(\mu)$ i intervallet $[a, \mu]$, då måste $f(x)$ anta värdet $0$ i detta intervall. Men $x = -1/2$ är det enda nollstället och ligger inte i intervallet. Antagandet att det finns ett sådant $\mu$ i intervallet leder till en motsägelse, alltså måste dess motsats vara sann; det finns inget sådant $\mu$ i intervallet, och då gälller att $f(x) > 0$ för alla $x$ i intervallet.
\\
\\På samma sätt visar vi att det för alla värden av $x$ intervallet $ ]-1/2, \infty[$ gäller att $f(x) < 0$ eftersomm $b$ ligger i detta intervall och $f(b) < 0$.

\centerline{$\qed$}

\end{document}
