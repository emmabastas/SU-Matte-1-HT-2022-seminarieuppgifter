\documentclass{article}

\usepackage{amsmath}
\usepackage{amsthm}
\usepackage{mathtools}

\usepackage{mdframed}
\usepackage{lipsum}

\newmdtheoremenv{lemma}{Theorem}
\renewcommand{\thefootnote}{\fnsymbol{footnote}}

\title{Seminarieuppgift 4 - Absolutbelopp}
\author{Emma Bastås}

\begin{document}

\maketitle

\noindent Uppgiften är att finna de värden på $x$ som upfyller olikheten:

\begin{gather*}
  2|x+2| < |x| + |x+3|\text{.} \label{*}\tag{$\star$}
\end{gather*}

\noindent Vi adderar $2|x+2|$ till båda led och skriver sedan om uttrycket med hjälp av identiteten $|x| = \sqrt{x^{2}}$:

\begin{align*}
  \text{(\ref{*})} \iff& \quad 0 < |x| + |x + 3| - 2|x+2| \\
  \iff& \quad 0 < \sqrt{x^{2}} + \sqrt{(x + 3)^{2}} - 2 \sqrt{(x + 2)^{2}}\text{.} \label{**}\tag{$\star\star$}
\end{align*}
\noindent Låt $f(x)$ definieras till högerledet i uttrycket (\ref{**}):

\begin{gather*}
  f(x) =  \sqrt{x^{2}} + \sqrt{(x + 3)^{2}} - 2 \sqrt{(x + 2)^{2}}\text{.}
\end{gather*}

\noindent Uppgiften omformuleras nu till denna ekvivalenta problemställning:
\\
\\
\centerline{\emph{Finn de värden på $x$ som upfyller olikheten: $f(x) > 0$.}}
\\
\\
Nu finner vi alla nollställen till $f(x)$, vi börjar med att ställa upp ekvationen och flytta över en av rötterna till högerledet:

\begin{align}
  \quad & f(x) = 0 \nonumber\\
  \iff \quad &\sqrt{x^{2}} + \sqrt{(x + 3)^{2}} - 2 \sqrt{(x + 2)^{2}} = 0 \nonumber\\
  \iff \quad &\sqrt{x^{2}} + \sqrt{(x + 3)^{2}} = 2 \sqrt{(x + 2)^{2}}\text{.} \label{initial}
\end{align}
\\
Båda led kvadreras varefter vi förenklar så att de kvarvarande kvadratrötterna står ensamma i högerledet:

\begin{align}
  \text{(\ref{initial})} \implies \quad & \biggl(\sqrt{x^{2}} + \sqrt{(x + 3)^{2}}\biggr)^{2} = \biggl(2 \sqrt{(x + 2)^{2}}\biggr)^{2}\text{.} \nonumber\\
  \iff \quad & x^{2} + 2\sqrt{x^{2}}\sqrt{(x+3)^{2}} + (x + 3)^{2} = 4(x+2)^{2} \nonumber\\
  \iff \quad & 2x^{2} + 10x + 7 = 2\sqrt{(x(x+3))^{2}}\text{.} \label{squared}
\end{align}
\\
Vi låter $y = x(x+3)$ och substituerar in det it (\ref{squared}):

\begin{gather}
  \text{(\ref{squared})} \iff \quad 2x^2 + 10x + 7 = 2\sqrt{y^2}\text{.} \label{with_y}
\end{gather}
\\
Vi noterar att beroende på $x$ är antingen $y \geq 0$ varpå $\sqrt{y^{2}} = y$ eller så är $y < 0$ varpå $\sqrt{y^{2}} = -y$. Det ger två fall för (\ref{with_y}):

\begin{align}
  \text{(\ref{with_y})} \iff
  \begin{cases}
    2x^{2} + 10x + 7 = 2y, & \text{då}\; y \geq 0 \\
    2x^{2} + 10x + 7 = -2y, & \text{då}\; y < 0
  \end{cases}\text{.} \label{cases}
\end{align}
\\
Plockar vi bort vilkoren från (\ref{cases}) implicerar detta en vanlig första- respektive andragradsekvaion med lösningarna $x = -\tfrac{7}{2}$, $x = -\tfrac{7}{4}$ och $x = -\tfrac{1}{2}$. (\ref{initial}) har medfört dessa lösningar och vi testar vilka som också löser (\ref{initial}) medelst insättning och finner att $x = -\tfrac{1}{2}$ är den enda lösningen.\footnote{Om du har en gnagande oro över att vi missat någon lösning så prova att kvadrera båda led i (\ref{squared}). Detta ger en tredjegradsekvation och de tre lösningar vi fann här löser också den ekvationen. Varför det är på detta sätt förstår jag inte, men det kan väll inte vara en slump!?}
\\
\\
Nu till slutklämmen:
\\
\\
$f(x)$ är en kontinuerlig funktion och $x = -\tfrac{1}{2}$ är dess enda nollställe. Talet $-2$ ligger i intervallet $\left]-\infty, -\tfrac{1}{2}\right[$ och $f(-2) > 0$. Antag ett $\mu$ i detta intervall så att $f(\mu) < 0$. Enligt sats (14) i Persson och Böiers \emph{Analys i en variabel}\footnote{Upplaga 3:2 s.153} medför detta att $f(x)$ antar alla värden mellan $f(-2)$ och $f(\mu)$ i intervallet $\left[-2, \mu\right]$. Då kommer $f(x)$ anta värdet $0$. Men $x = -\tfrac{1}{2}$ är det enda nollstället till $f(x)$ och ligger inte i intervallet. Antagandet att $\mu$ finns har producerat en motsägelse, alltså måste motsatsen vara sann; det finns inget $\mu$ i intervallet och då gäller $f(x) > 0$ för alla $x$ i intervallet.
\\
\\På samma sätt visar vi att det för alla värden av $x$ intervallet $ \left]-1/2, \infty\right[$ gäller att $f(x) < 0$ eftersomm talet $0$ ligger i detta intervall och $f(0) < 0$.

\centerline{$\qed$}

\end{document}
