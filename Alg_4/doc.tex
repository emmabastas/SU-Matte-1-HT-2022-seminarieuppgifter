\documentclass{article}

\usepackage{amsmath}
\usepackage{amsthm}
\usepackage{mathtools}
\usepackage{systeme}
\usepackage{lipsum}
\usepackage{gauss}

\renewcommand{\thefootnote}{\fnsymbol{footnote}}
\renewcommand{\rowmultlabel}[1]{\cdot \; #1}

\title{Algebrauppgift 4 - Ekvationssystem}
\author{Emma Bastås}
\date{Oktober 23, 2022}

\begin{document}

\maketitle

\noindent Uppgiften är att till varje reellt tal $a$ bestämma antalet lösningar till följande ekvationssystem:

\begin{gather*}
  \begin{cases}
  (4 - a)x_{1} + 2x_{2} - x_{3} = 1\\
  -2x_{1} + (1 - a)x_{2} + 2x_{3} = -2\\
  -x_{1} + 2x_{2} + (4 - a)x_{3} = 1
  \end{cases}\text{.} \tag{$\star$}\label{*}
\end{gather*}
\\
Först noterar vi att alla tre ekvationer är linjära, och vi kan därför skriva om ekvationssystem till denna ekvivalenta matrisekvation:

\begin{gather*}
  \text{(\ref{*})} \iff AX = B
\end{gather*}
\\
där:
\begin{gather*}
  A =
  \begin{pmatrix}
    4 - a & 2 & - 1 \\
    -2 & 1 - a & 2 \\
    -1 & 2 & 4 - a
  \end{pmatrix}\quad
  X =
  \begin{pmatrix}
    x_{1} \\
    x_{2} \\
    x_{3}
  \end{pmatrix}\quad
  B =
  \begin{pmatrix}
    1 \\
    -2 \\
    1
  \end{pmatrix}\text{.}
\end{gather*}
\\
För linjära matrisekvationer på denna form finns precis tre möjligheter avseende lösningar:
\begin{itemize}
  \item Det finns ingen lösning.
  \item Det finns precis en lösning.
  \item Det finns oöndligt månda lösningar.
\end{itemize}

\noindent Att determinanten till $A$ är skiljd från noll är ekvivalent med att $A$ har en invers, vilket i sin tur är ekvivalent med att $AX = B$ har precis en lösning (nämligen $X = A^{-1}B$). Är determinanten noll så gäller då antingen att ingen eller oändligt många lösningar till $AX = B$ finns.
\\
\\
Vi finner nu de värden på $a$ för vilket $\det A = 0$. Här använder vi oss helt enkelt av den otympliga formeln för determinanter för $3\times3$ matriser, $\det (a_{ij}) = a_{11} a_{22} a_{33} + a_{12} a_{23} a_{31} + a_{13} a_{21} a_{32} - a_{11} a_{23} a_{32} - a_{12} a_{21} a_{33} - a_{13} a_{22} a_{31} $:\footnote{Sarrus regel eller cofaktorexpansion går såklart lika bra.}

\begin{align}
  &\det A = 0\nonumber\\
  \iff \quad &(4 - a)(1 - a)(4 - a) + 2\cdot2\cdot(-1) + (-1)\cdot(-2)\cdot2\nonumber\\
  &- (4 - a)\cdot2\cdot2 - 2\cdot(-2)\cdot(4 - a) - (-1)\cdot(1 - a)\cdot(-1) = 0\nonumber\\
  \iff \quad &-a^{3} + 9a^{2} -23a +15 = 0\text{.} \label{3rd_degree}
\end{align}
\\
Vi finner att $\det A = 0$ är en ekvivalent ekvation till (\ref{3rd_degree}). Att lösa tredjegradsekvationer i det generella fallet ligger utanför ramen för denna kurs, men vi har tur (eller snarare ett tillrättalagt problem) och kan med hjälp av Rationella rotsatsen, faktorsatsen och polynomdivision lösa ekvationen.\footnote{Se kapitell 9 och 10 i algebrkompendiumet för en redogörelse för hur detta görs.}
\\
\\
Vi finner att lösningarna till $\det A = 0$ är $a = 1$, $a = 3$ och $a = 5$. Detta är ekvivalent med att $AX = B$ har oändligt många, eller inga lösningar för var och ett av dessa värden på $a$. För alla andra värden på $a$ gäller därav en unik lösning till ekvationen.
\\
\\
Nu återstår att bestämma antalet lösningar för $a = 1$, $a = 3$ och $a = 5$.
\\
\\
För dessa tre fall finns alltså precis två alternativ: Inga eller oändligt många lösningar. $AX = B$ har oändligt många lösningar om och endast om det finns en $3 \times 3$ matris $M$ så att:

\begin{gather*}
  MAX = B \\
  \iff
  \begin{pmatrix}
    i & j & k
  \end{pmatrix}
  X =
  \begin{pmatrix}
  \begin{pmatrix}
\end{gather*}
\\
\\
\textbf{Fallet $a = 1$:}
\\
\\
Vi låter $a = 1$, ställer upp ekvationen $AX = B$ för Gauss-Jordan eliminering och utför 2 operationer:

\begin{equation*}
  \begin{gmatrix}[p]
    3 & 2 & -1 & \bigm| & 1 \\
    -2 & 0 & 2 & \bigm| & -2 \\
    -1 & 2 & 3 & \bigm| & 1
    \rowops
      \add[-1]{0}{2}
      \add[-2]{1}{2}
  \end{gmatrix}
  \rightarrow
  \begin{pmatrix}
    3 & 2 & -1 & \bigm| & 1 \\
    -2 & 0 & 2 & \bigm| & -2 \\
    0 & 0 & 0 & \bigm| & 4
  \end{pmatrix}\text{.}
\end{equation*}
\\
Skrivsättet här är bara lite smidig notation, det som egentligen sker är följande:

\begin{gather}
  AX = B \nonumber \\[2pt]
  \iff \quad
               \begin{pmatrix}
                 1 & & \\
                 & 1 & \\
                 & -2 & 1
               \end{pmatrix}
               \begin{pmatrix}
                 1 & & \\
                 & 1 & \\
                 -1 & & 1
               \end{pmatrix}
               AX =
                              \begin{pmatrix}
                 1 & & \\
                 & 1 & \\
                 -1 & & 1
               \end{pmatrix}
               \begin{pmatrix}
                 1 & & \\
                 & 1 & \\
                 & -2 & 1
               \end{pmatrix}
               B \nonumber \\[4pt]
  \iff \quad
               \begin{pmatrix}
                 3 & 2 & -1 \\
                 -2 & 0 & 2 \\
                 0 & 0 & 0
               \end{pmatrix}
               X =
               \begin{pmatrix}
                 1 \\ -2 \\ 4
               \end{pmatrix}
               \label{a=1}
\end{gather}
\\
Detta är ekvivalent med ekvatoinssystemet:

\begin{gather*}
  \begin{cases}
    \cdots = 1 \\
    \cdots = -2 \\
    0x_{0} + 0x_{1} + 0x_{2} = 4
  \end{cases}
\end{gather*}
\\
och det är uppenbar att detta saknar lösningar. Alltså saknar den ekvivalenta ekvationen $AX = B$ lösningar för $a = 1$.
\\
\\
\textbf{Fallet a = 3:}
\\
\\
Vi låter $a = 3$ och betraktar $AX = B$:

\begin{gather*}
  \begin{pmatrix}
    1 &
  \end{pmatrix}
\end{gather*}

\end{document}
