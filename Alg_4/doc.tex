\documentclass{article}

\usepackage{amsmath}
\usepackage{amsthm}
\usepackage{mathtools}
\usepackage{systeme}
\usepackage{lipsum}

\renewcommand{\thefootnote}{\fnsymbol{footnote}}

\title{Seminarieuppgift 6 - Lösningar till ekvationssystem}
\author{Emma Bastås}
\date{Oktober 9, 2022}

\begin{document}

\maketitle

\noindent Uppgiften är att till varje reellt tal $a$ bestämma antalet lösningar till följande ekvationssystem:

\begin{gather*}
  \begin{cases}
  (4 - a)x_{1} + 2x_{2} - x_{3} = 1\\
  -2x_{1} + (1 - a)x_{2} + 2x_{3} = -2\\
  -x_{1} + 2x_{2} + (4 - a)x_{3} = 1
  \end{cases}\text{.} \tag{$\star$}\label{*}
\end{gather*}
\\
Först noterar vi att alla tre ekvationer är linjära, och vi kan därför skriva om ekvationssystem till denna ekvivalenta matrisekvation:

\begin{gather*}
  \text{(\ref{*})} \iff AX = B\text{.}
\end{gather*}

där:
\begin{gather*}
  A =
  \begin{pmatrix}
    4 - a & 2 & - 1 \\
    -2 & 1 - a & 2 \\
    -1 & 2 & 4 - a
  \end{pmatrix}\quad
  X =
  \begin{pmatrix}
    x_{1} \\
    x_{2} \\
    x_{3}
  \end{pmatrix}\quad
  B =
  \begin{pmatrix}
    1 \\
    -2 \\
    1
  \end{pmatrix}\text{.}
\end{gather*}
\\
För linjära matrisekvationer på denna form finns precis tre möjligheter avseende lösningar:
\begin{itemize}
  \item Det finns ingen lösning.
  \item Det finns precis en lösning.
  \item Det finns oöndligt månda lösningar.
\end{itemize}

\noindent Att determinanten till $A$ är skiljd från noll är ekvivalent med att $A$ har en invers, vilket i sin tur är ekvivalent med att $AX = B$ har precis en lösning (nämligen $X = A^{-1}B$). Är determinanten noll så gäller då antingen att ingen eller oändligt många lösningar till $AX = B$ finns.
\\
\\
Vi finner nu de värden på $a$ för vilket $\det A = 0$. Här använder vi oss helt enkelt av den otympliga identiteten $det A = a_{11} a_{22} a_{33} + a_{12} a_{23} a_{31} + a_{13} a_{21} a_{32} - a_{11} a_{23} a_{32} - a_{12} a_{21} a_{33} - a_{13} a_{22} a_{31} $. Men det går lika bra att använda annan en metod, t.ex Sarrus regel eller cofaktorexpansion.

\begin{align}
  &\det A = 0\nonumber\\
  \iff \quad &(4 - a)(1 - a)(4 - a) + 2\cdot2\cdot(-1) + (-1)\cdot(-2)\cdot2\nonumber\\
  &- (4 - a)\cdot2\cdot2 - 2\cdot(-2)\cdot(4 - a) - (-1)\cdot(1 - a)\cdot(-1) = 0\nonumber\\
  \iff \quad &-a^{3} + 9a^{2} -23a +15 = 0\text{.} \label{3rd_degree}
\end{align}

Att lösa tredjegradsekvationer ligger utanför denna seminariekurs, men rationella rotsatsen ger oss att alla lösningarna till (\ref{3rd_degree}) är något av talen $\pm1, \cdots, \pm 15$. 30 potentiella lösningar är tillräckligt få för att det ska vara görbart att testa var och en av dom. Vi gör detta och finner en lösning redan vid första försöket! $a = 1$. Nu kan vi medelst faktorssatsen och polynomdivision skriva om till (\ref{3rd_degree}) till den ekvivalenta ekvationen $(a-1)(-a^{2} + 8a - 5) = 0$ och vi finner de två resterande lösningarna $a = 3$ och $a = 5$.
\\
\\
Nu har vi alltså visat att $\det A = 0$ om och endast om $a = 1$, $a = 3$ eller $a = 5$, vilket är ett ekvivalent påstående med att $AX = B$ har oöndligt många, eller inga lösningar för var och ett av dessa värden på $a$. Försöket alla andra värden på $a$ gäller därav en unik lösning till $AX = B$.
\\
\\
Nu återstår att bestämma antalet lösningar för $a = 1$, $a = 3$ och $a = 5$.
\\
\\
\textbf{Fallet $a = 1$:}
\\
\\
Vi låter $a = 1$ och betraktar (\ref{*})

\begin{gather}
  \text{(\ref{*})} \quad \stackrel{a = 1}{\iff} \quad
  \systeme{
  3x_{1} + 2x_{2} - x_{3} = 1,
  -2x_{1} + 2x_{3} = -2,
  -x_{1} + 2x_{2} + 3x_{3} = 1 \text{.}
  } \label{a=1}
\end{gather}
\\
Den första och sista ekvationen i (\ref{a=1}) har samma högerled och vi kan då skriva om dom till denna ekvivalenta ekvation:

\begin{align*}
  \quad & 3_{x1} + 2x_{2} - x_{3} = -x_{1} + 2x_{2} + 3x_{3} \\
  \iff \quad & 3x_{1} - x_{3} = -x_{1} + 3x_{3} \\
  \iff \quad & 4x_{1} = 4x_{3} \\
  \iff \quad & x_{1} = x_{3}
\end{align*}
\\
Den mittersta ekvationen i (\ref{a=1}) är ekvivalent med $x_{1} = x_{3} + 1$. Vi har nu ett nytt ekvationssystem som är ekvivalent med (\ref{a=1}):

\begin{gather*}
  (\ref{a=1}) \quad \iff \quad
  \systeme{
  x_{1} = x_{3},
  x_{1} = x_{3} + 1
  }
\end{gather*}
\\
vilket inte har några lösningar. Alltså har vi visat att (\ref{*}) saknar lösningar då $a = 1$.

\end{document}
