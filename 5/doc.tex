\documentclass{article}

\usepackage{amsmath}
\usepackage{amsthm}
\usepackage{mathtools}

\usepackage{mdframed}
\usepackage{lipsum}

\newmdtheoremenv{lemma}{Theorem}
\renewcommand{\thefootnote}{\fnsymbol{footnote}}

\title{Seminarieuppgift 5 - Tangenter genom punkt}
\author{Emma Bastås}

\begin{document}

\maketitle

\noindent Uppgiften är att finna de tangenter till $y = f(x)$ (om det finns några) som går genom punkten $P$ där $f(x) = 2x^{3} - 3x + 5$ och $P = (0, 1)$.
\\
\\
Persson och Böiers definierar\footnote{\emph{Analys i en variabel,} Upplaga 3:2 s.189 \textbf{Geometrisk tolkning av derivata.}} tangenten till funktionskurvan $y = f(x)$ i punkten $(x_{0}, f(x_{0}))$ som den linje vars ekvation är:

\begin{gather*}
  y = f'(x_{0})(x - x_{0}) + f(x_{0})\text{.} \label{*}\tag{$\star$}
\end{gather*}
\\
Att en linje $y = g(x)$ går genom en punkt $I$ är ett ekvivalent påstående med att $I_{y} = g(I_{x})$. För oss som är intresserade av de fall då linjen (\ref{*}) går genom punkten $P$ finnes dessa då fall likheten:

\begin{gather*}
  P_{y} = f'(x_{0})(P_{x} - x_{0}) + f(x_{0}) \label{**}\tag{$\star\star$}
\end{gather*}
\\
gäller. Här är $x_{0}$ den enda obekanta, vi expanderar $P_{y}$, $P_{x}$, $f'(x_{0})$ och $f(x_{0})$ och löser ekvationen:

\begin{align*}
  \text{(\ref{**})} \iff \quad & 1 = (6x_{0}^{3})(0 - x_{0}) + 2x_{0}^{3} - 3x_{0} + 5 \\
  \iff \quad & x_{0}^{3} = 1 \\
  \iff \quad & x_{0} = 1\text{.}
\end{align*}
\\
Ekvationen (\ref{**}) har alltså $x_{0} = 1$ som enda lösning. Detta innebär att (\ref{*}) går genom $P$ då $x_{0} = 1$. Sätter vi in $x_{0} = 1$ i (\ref{*}) och förenklar får vi linjen:

\begin{align*}
  & y = f'(1)(x - 1) + f(1) \\
  \iff \quad & y = (6 - 3)(x - 1) + 2 - 3 + 5 \\
  \iff \quad & y = 3x + 1
\end{align*}
\\
Som alltså är den enda tangenten till $f(x)$ som går igenom $P$.

\centerline{$\qed$}

\end{document}
