\documentclass{article}

\usepackage{amsmath}
\usepackage{amsthm}
\usepackage{mathtools}

\usepackage{mdframed}
\usepackage{lipsum}

\newmdtheoremenv{lemma}{Theorem}
\renewcommand{\thefootnote}{\fnsymbol{footnote}}

\title{Seminarieuppgift 5 - Tangenter genom punkt}
\author{Emma Bastås}
\date{Oktober 9, 2022}

\begin{document}

\maketitle

\noindent Uppgiften är att finna de tangenter (om det finns några) till $y = f(x)$ som går genom punkten $P$ där $f(x) = 2x^{3} - 3x + 5$ och $P = (0, 1)$. \\
\\
Persson och Böiers definition av tangenter\footnote{\emph{Analys i en variabel,} Upplaga 3:2 s.189 \textbf{Geometrisk tolkning av derivata.}} ger att tangenten till $y = f(x)$ i punkten $(x_{0}, f(x_{0}))$ är den räta linje som uppfyller ekvationen:

\begin{gather}
  y - f(x_{0}) = f'(x_{0})(x - x_{0})\text{.} \label{orig}
\end{gather}
\\
Persson och Böiers anger också att en rät linje (som inte är parallell med $y$-axeln) och som går genom $(0, 1)$ ges av ekvationen\footnote{\emph{Analys i en variabel}, Upplaga 3:2 s.25.}:

\begin{gather}
  y = kx + 1\text{.} \label{enpunktsform}
\end{gather}
\\
Detta ger att alla värden på $x_{0}$ för vilket (\ref{orig}) är ekvivalent med (\ref{enpunktsform}) ger en rät linje som är både tangent till $y = f(x)$ och går genom $P$, vilket är vad vi söker.
\\
\\
Vi expanderar $f(x_{0})$ och $f'(x_{0})$ i (\ref{orig}) och omarbetar ekvationen till att vara på samma form som i (\ref{enpunktsform}):

\begin{align}
  \text{(\ref{orig})} \iff \quad & y - (2x_{0}^{3} - 3x_{0} + 5) = (6x_{0}^{2} - 3)(x - x_{0}) \nonumber \\
  \iff \quad & y = x(6x_{0}^{2} - 3) - 4x_{0}^{3} + 5\text{.} \label{processed}
\end{align}
\\
Nu ser vi att (\ref{processed}) är ekvivalent med (\ref{enpunktsform}) då $-4x_{0}^{3} + 5 = 1$ vilket inträffar då $x_{0} = 1$. Vi sätter in $x_{0} = 1$ i (\ref{processed}) och erhåller den räta linjen:

\begin{gather*}
  y = 3x + 1 \label{solution}
\end{gather*}
\\
som vi visat är den enda tangenten till $y = f(x)$ som går genom $P$. $\qed$

\thispagestyle{empty}

\end{document}
