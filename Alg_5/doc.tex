\documentclass{article}

\usepackage{amsmath}
\usepackage{amsthm}
\usepackage{mathtools}
\usepackage{systeme}
\usepackage{lipsum}

\renewcommand{\thefootnote}{\fnsymbol{footnote}}

\title{Algebrauppgift 5  - Rekursiv talföljd}
\author{Emma Bastås}
\date{Oktober 23}

\begin{document}

\maketitle


\thispagestyle{empty}

\noindent Uppgiften är att med induktion finna en icke-rekursiv formel till följande talföljd:

\begin{gather*}
  a_{1} = 1, \quad
  a_{2} = 3 \\
  a_{n+1} = 2a_{n} - a_{n-1} \; n \geq 2\text{.}
\end{gather*}
\\
Vi börjar med att beräkna $a_{n}$ för ett litet antal värden på $n$ i hoppet om att finna ett mönster:

\begin{align*}
  a_{3} = 2a_{2} - a_{1} = 2\cdot3 - 1 &= 5 \\
  a_{4} = 2a_{3} - a_{2} = 2\cdot5 - 3 &= 7 \\
  a_{5} = 2a_{4} - a_{3} = 2\cdot7 - 5 &= 9\text{.}
\end{align*}
\\
Vi ser ett mönster! För $n \leq 5$ beskrivs talföljden $a_{n}$ av formeln:

\begin{gather*}
  a_{n} = 2n - 1 \tag{$\star$}\label{*} \\[10pt]
  a_{1} = 2\cdot1 - 1 = 1\\[-6pt]
  \vdots \\[-2pt]
  a_{5} = 2\cdot5 - 1 = 9\text{.}
\end{gather*}
\\
Kanske gäller (\ref{*}) även för $n > 5$, kanske inte.. Antag nu att (\ref{*}) gäller för något $n$ och $n-1$. Då gäller (\ref{*}) även för $n+1$, ty:

\begin{align*}
  a_{n+1} &= 2a_{n} - a_{n - 1} \\
          &= 2(2n - 1) \;-\; (2(n - 1) - 1) \\
          &= 4n - 2 \;-\; 2n + 2 + 1 \\
          &= 2n + 1 \\
  &= 2(n+1) - 1\text{.}
\end{align*}
\\
Vi har alltså visat att (\ref{*}) gäller för b.la. $n = 1$ och $n = 2$, samt att om (\ref{*}) gäller för något $n$ och $n-1$ så gäller (\ref{*}) även för $n+1$. Enligt induktionsprincipen gäller då (\ref{*}) för alla $n$.
\\[8pt]
\centerline{$\qed$}

\thispagestyle{empty}

\end{document}
