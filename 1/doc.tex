\documentclass{article}

\usepackage{amsmath}
\usepackage{amsthm}
\usepackage{mathtools}

\title{Seminarieuppgift 1 - Rotekvationer}
\author{Emma Bastås}

\begin{document}

\maketitle

Uppgiften är att finna alla reella lösningar till

\begin{gather*}
  \sqrt{x + 3} + \sqrt{2 - x} = 3 \label{orig}\tag{0}.
\end{gather*}

\section*{Algebraisk bearbetning}

Vi börjar med att flytta över ena roten till högerledet för att sedan kvadrera:

\begin{align}
  (\text{\ref{orig}})\quad &\Leftrightarrow  &\sqrt{x + 3}   &= 3 - \sqrt{2 - x} \nonumber &\\
  &\Rightarrow      &x + 3          &= 9 - 6\sqrt{2 - x} + 2 - x & \label{kvad1}\\
  &\Leftrightarrow  &4 - x          &= 3\sqrt{2 - x}\text{.} \label{noroot} &
\end{align}

Återigen kvadrerar vi båda led:

\begin{align}
  (\text{\ref{noroot}})\quad &\Rightarrow &(4 - x)^2 &= (3\sqrt{2 - x})^2 & \label{kvad2}\\
  &\Leftrightarrow &16 - 8x + x^2 &= 9(2 - x) \nonumber&\\
  &\Leftrightarrow &x^2+ x &= 2\text{.} &  \label{2pol}
\end{align}

Vi ser att ekvation (\ref{2pol}) är ett andragradspolynom som går att lösa med b.la. kvadratkompletering:

\begin{align}
  (\text{\ref{2pol}})\quad &\Leftrightarrow &\biggl(x + \frac{1}{2}\biggr)^2 - &\biggl(\frac{1}{2}\biggr)^2 = 2 \nonumber &\\[6pt]
  &\Leftrightarrow &\biggl(x + \frac{1}{2}&\biggr)^2 = \frac{9}{4}\text{.} & \label{kvadrarkompleterad}
\end{align}

Vi drar kvadratroten hur båda led:

\begin{align}
  (\text{\ref{kvadrarkompleterad}})\quad &\Leftrightarrow  && x + \frac{1}{2} = \pm \sqrt{\frac{9}{4}} & \nonumber \\[10pt]
  &\Leftrightarrow  && x = \frac{ \pm 3 - 1 }{ 2 } \nonumber &
\end{align}

och ser att $x = -2$ och $x = 1$ löser ekvation (\ref{2pol}).

\section*{Verifiering av lösningar}

$x = -2$ och $x = 1$ löser ekvation (\ref{2pol}), men inte nödvändigtvis vår ursprungliga ekvation (\ref{orig}). På två platser, rad (\ref{kvad1}) och (\ref{kvad2}) har vi kvadrerat båda led vilket har medfört ekv (\ref{2pol}), men denna ekvation är inte nödvändigtvis ekvivalent med den ursprungliga. Mer konkret kan vi säga att lösningarna till ekv (\ref{orig}) är en delmängd till lösningarna för ekv (\ref{2pol}). I detta fall då ekv (\ref{2pol}) endast har två lösningar kan vi helt enkelt testa båda i ekv (\ref{orig}).

\begin{align*}
  & \text{($x = -2$)}&  \sqrt{(-2) + 3} &+ \sqrt{2 - (-2)} &= \quad 1 + 2 \quad = \quad 3 \quad\quad\quad&&\\
  & \text{($x = 1$)}&  \sqrt{1 + 3}    &+ \sqrt{2 - 1}    &= \quad 2 + 1 \quad =    \quad 3 \text{.}    \quad\quad\quad&&
\end{align*}

Därmed ser vi att $x = -2$ och $x = 1$ är lösningarna till $\sqrt{x + 3} + \sqrt{2 - x} = 3$.

\centerline{$\qed$}

\end{document}
