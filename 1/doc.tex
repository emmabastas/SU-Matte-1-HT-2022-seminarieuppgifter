\documentclass{article}

\usepackage{amsmath}
\usepackage{amsthm}
\usepackage{mathtools}

\title{Seminarieuppgift 1 - Rotekvationer}
\author{Emma Bastås}

\begin{document}

\maketitle

Uppgiften är att finna alla reella lösningar till

\begin{gather*}
  \sqrt{x + 3} + \sqrt{2 - x} = 3 \label{orig}\tag{0}
\end{gather*}

\section*{Bortkvadrering av rötter}

Vi börjar med att flytta över ena roten till högerledet för att sedan kvadrera

\begin{align}
  &\Leftrightarrow  &\sqrt{x + 3}   &= 3 - \sqrt{2 - x} &\\
  &\Rightarrow      &x + 3          &= 9 - 6\sqrt{2 - x} + 2 - x & \label{kvad1}\\
  &\Leftrightarrow  &4 - x          &= 3\sqrt{2 - x} &
\end{align}

Återigen kvadrerar vi båda led

\begin{align}
  &\Rightarrow &(4 - x)^2 &= (3\sqrt{2 - x})^2 & \label{kvad2}\\
  &\Leftrightarrow &16 - 8x + x^2 &= 9(2 - x) &\\
  &\Leftrightarrow &x^2+ x &= 2 &  \label{2pol}
\end{align}

\section*{Kvadratkompletering och lösningar}

Vi ser att ekvation (\ref{2pol}) är ett andragradspolynom som går att lösa med b.la. kvadratkompletering.

\begin{align}
  & &x^2 + x &= 2 &\\[6pt]
  &\Leftrightarrow &\biggl(x + \frac{1}{2}\biggr)^2 - &\biggl(\frac{1}{2}\biggr)^2 = 2 &\\[6pt]
  &\Leftrightarrow &\biggl(x + \frac{1}{2}&\biggr)^2 = \frac{9}{4} &
\end{align}
Vi drar kvadratroten hur båda led

\begin{align}
  &\Leftrightarrow  && x + \frac{1}{2} = \pm \sqrt{\frac{9}{4}} & \\[10pt]
  &\Leftrightarrow  && x = \frac{ \pm 3 - 1 }{ 2 } &
\end{align}

och ser att $x = -2$ och $x = 1$ löser ekvation (\ref{2pol}).

\section*{Verifiering av lösningar}

$x = -2$ och $x = 1$ löser ekvation (\ref{2pol}), men inte nödvändigtvis vår ursprungliga ekvation (\ref{orig}). På två platser, rad (\ref{kvad1}) och (\ref{kvad2}) har vi kvadrerat båda led vilket har medfört ekv (\ref{2pol}), men denna ekvation är inte ekvivalent med den ursprungliga. Mer konkret kan vi säga att lösningarna till ekv (\ref{orig}) är en delmängd till lösningarna för ekv (\ref{2pol}). I detta fall då ekv (\ref{2pol}) endast har två lösningar kan vi helt enkelt testa båda i ekv (\ref{orig}).

\begin{align*}
  & \text{($x = -2$)}&  \sqrt{(-2) + 3} &+ \sqrt{2 - (-2)} &= \quad 1 + 0 \quad \neq \quad 3 \quad\quad\quad&&\\
  & \text{($x = 1$)}&  \sqrt{1 + 3}    &+ \sqrt{2 - 1}    &= \quad 2 + 1 \quad =    \quad 3    \quad\quad\quad&&
\end{align*}

Nu ser vi att $x = 1$ är den enda lösningen till $\sqrt{x + 3} + \sqrt{2 - x} = 3$.

\centerline{$\qed$}

\end{document}
