\documentclass{article}

\usepackage{amsmath}
\usepackage{amsthm}
\usepackage{mathtools}
\usepackage{hyperref}

\title{Seminarieuppgift 2 - Diofantiska Ekvationer}
\author{Emma Bastås}

\begin{document}

\maketitle

Uppgiften är att finna samtliga positiva lösningar till följande diofantiska ekvation:

\begin{gather*}
  29x + 43y = 4000. \label{*}\tag{$\star$}
\end{gather*}

\section*{Referenser}

Denna text använder sig av satser från som finnes i \emph{Algebra I, tionde tryckningen} av \emph{Bøgvad}, \emph{Xantcha} och \emph{Granath}.

Kompendiet finns som PDF på: \url{kurser.math.su.se/pluginfile.php/143626/mod_resource/content/12/Algebra-i-tionde-tryckningen.pdf}

\section*{Heltalslösningar}

Enligt lemma 5.13 är $SGD(29, 43)\: |\: 4000$ ett nödvändigt villkor för att ekv (\ref{*}) ska ha lösningar. Därför testar vi om detta villkor är uppfylt. Euklides algoritm (sats 5.10) används för att beräkna $SGD(29, 43)$ och alla steg i algoritmen skrivs ut då detta kommer visa sig vara av användning alldeles strax.

\begin{align}
  &43& &=& &1& &\cdot& &29& &+& 14&& \label{sgd-1} \\
  &29& &=& &2& &\cdot& &14& &+& 1\text{.} && \label{sgd-2}
\end{align}

Av detta följer att $SGD(29, 43) = 1$ vilket delar $4000$ och vi kan således fortsätta sökandet efter lösningar.

Nu betraktar vi en ekvation som liknar ekv (\ref{*}) men med högerled $1$ istället för $4000$:

\begin{gather*}
  29x + 43y = 1\text{.} \label{H}\tag{H}
\end{gather*}

För att finna en \emph{partikulärlösning} till denna ekvation så använder vi ekvationerna från tidigare beräkning av $SGD(29, 43)$ med Euklides algoritm, fast omskrivet så att resterna står ensamma i vänsterledet.

\begin {align*}
  (\text{\ref{sgd-1}}) &\iff &14 &= 43 - 29 \label{sgd-1*}\tag{\ref{sgd-1}*}\\
  (\text{\ref{sgd-2}}) &\iff &1 &= 29 - 2 \cdot 14\text{.} \label{sgd-2*}\tag{\ref{sgd-2}*}
\end{align*}

Vi sätter in ekv (\ref{sgd-1*}) i ekv (\ref{sgd-2*}) och får:

\begin{gather*}
  1 = 29 - 2 \cdot (43 - 29)\text{,}
\end{gather*}

vilket förenklas till:

\begin{gather}
  29 \cdot 3 + 43 \cdot (-2) = 1\text{.} \label{partikularlosning_till_H}
\end{gather}

Nu ser vi att $x = 3$ och $y = (-2)$ är \emph{en} lösning till ekv (\ref{H}). Vi multiplicerar båda led med $4000$:

\begin{align*}
  \text{(\ref{partikularlosning_till_H})} &\iff &4000(29 \cdot 3 + 43 \cdot (-2)) = 4000 \\
  &\iff &29 \cdot 12000 + 43 \cdot (-8000) = 4000
\end{align*}

och har en lösning till (\ref{*}); $x = 12000$ och $y = -8000$.

Sats 5.14 säger att vi givet vår diofantiska ekvationen där $SGD(29, 43) = 1$ och med en partikulärlösning får den allmäna lösningen:

\begin{gather*}
  \begin{cases}
    x = 12000 - 43n \\
    y = -8000 + 29n
  \end{cases}
\end{gather*}

där $n$ är ett godtyckligt heltal.


\section*{Positiva heltalslösningar}

Att finna den allmäna lösningen till ekv (\ref{*}) var dock inte uppgiften, vi ska finna \emph{samtliga positiva lösningar}, alltså de lösningar där $x > 0$ och $y > 0$.

Om vi löser dessa två olikheter för $n$ så får vi:

För $x > 0$:

\begin{align*}
  & &0 < 12000 - 43n \\
  &\iff &43n < 43 \cdot 279 + 3 \\
  &\iff &n < 279 + 3 / 43 \label{XN}\text{.} \tag{XN}
\end{align*}

För $y > 0$:
\begin{align*}
  & &0 < (-8000) + 29n \\
  &\iff &29n > 29 \cdot 275 + 25 \\
  &\iff &n > 275 + 25 / 29\text{.} \label{YN}\tag{YN}
\end{align*}

Vi är dock bara intresserade heltal $n$, så olikheterna (\ref{XN}) och (\ref{YN}) skriver vi om och kombinerar till:

\begin{gather*}
  276 \leq n \leq 279\text{.} \label{N}\tag{N}
\end{gather*}

Nu är alltså påståendet att alla lösningar ska vara positiva ekvivalent med påstående (\ref{N}).

Eftersom detta endast ger oss fyra värden på $n$ -- $267$, $277$, $278$ och $279$ -- som upfyller olikheten så kan vi utan störra besvär beräkna motsvarande $x$ och $y$ värden som löser ekvationen. Vi finner då att alla positiva heltalslösningar till ekv (\ref{*}) är:

\begin{align*}
  &x = 132   &x=89 \\
  &y = 4     &y=33 \\
  \\
  &x = 46    &x=3 \\
  &y = 62    &y=91
\end{align*}

\centerline{$\qed$}

\end{document}
