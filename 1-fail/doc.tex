\documentclass{article}

\usepackage{amsmath}
\usepackage{amsthm}
\usepackage{mathtools}

\title{Semenarieuppgift 1 - försök 1}
\author{Emma Bastås}

\begin{document}

\maketitle

Uppgiften är att lösa ekvationen.

\begin{gather*}
  \sqrt{x + 3} + \sqrt{2 - x} = 3 \label{ekv:0}\tag{0}
\end{gather*}

Kvadrering av båda led medför följande ekvation och ekvivalenser

\begin{gather*}
  \sqrt{x + 3} + \sqrt{2 - x} = 3                      \\
  (\sqrt{x + 3} + \sqrt{2 - x})^{2} = 3^{2}             \\
  (x + 3) + 2\sqrt{x + 3}\sqrt{2 - x} + (2 - x) = 9    \\
  x - x + 3 + 2 + 2\sqrt{x + 3}\sqrt{2 - x} = 9        \\
  2\sqrt{x + 3}\sqrt{2 - x} = 4                        \\
  \sqrt{x + 3}\sqrt{2 - x} = 2                  \label{ekv:1}\tag{1}
\end{gather*}

Vänsterledet kan förenklas ytterligare

\begin{gather*}
  \sqrt{x + 3}\sqrt{2 - x} = \sqrt{(x + 3)(2 - x)} = \sqrt{2x + 6 - x^{2} - 3x} = \sqrt{-x^{2} - x + 6}
\end{gather*}

Vi kvadrerar återigen båda led vilket medför följande ekvation och ekvivalenser

\begin{gather*}
  (\sqrt{-x^{2} - x + 6})^{2} = 2^{2} \\
  -x^{2} - x + 6 = 4 \\
  -x^{2} - x + 2 = 0    \label{ekv:2}\tag{2}
\end{gather*}

Nu kvadratkompleterar vi vänsterledet

\begin{gather*}
  x^{2} + x + 2 = ((x - \frac{1}{2})^{2} - \frac{1}{4}) + 2 = (x - \frac{1}{2})^{2} + \frac{9}{4}
\end{gather*}

Den konstanta termen flyttas till högerledet och vi drar kvadratroten ur båda led, vilket medför följande ekvivalenser.

\begin{gather*}
  \sqrt{(x + \frac{1}{2})^{2}} = \pm \sqrt{\frac{9}{4}} \\
  x + \frac{1}{2} = \pm \frac{\sqrt{9}}{2}\\
  x = \frac{\pm 3 - 1}{2}        \label{ekv:3}\tag{3}
\end{gather*}

De två lösningar till denna ekvation (\ref{ekv:3}) är $x = 1$ och $x = - 2$.

Vår ursprungliga ekvation (\ref{ekv:0}) medör (\ref{ekv:1}), medför (\ref{ekv:2}), medför (\ref{ekv:3}) som har lösningarna ovan.
D.v.s alla lösningar till (\ref{ekv:0}) är också lösningar till (\ref{ekv:3}), men motsattsen gäller inte. Vi måste testa vilka lösningar till (\ref{ekv:3}) som också löser (\ref{ekv:0})




\end{document}
